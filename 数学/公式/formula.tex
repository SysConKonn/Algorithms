%formula.tex
\documentclass{article}
\usepackage{amsmath}
\usepackage{hyperref}
\usepackage{CJK}

\begin{document}

\title{\huge{FORMULA}}
\author{\emph{SysCon}}
\maketitle
\begin{center}
  \url{https://sysconkonn.github.io/}
\end{center}

\newpage

\begin{CJK}{UTF8}{gbsn}
  \begin{itemize}
    %\Large
    \section{斯特林公式}
    \subsection{式子:}\\
    $$ n! \approx \sqrt{2\pi n} (\frac{n}{e})^n $$\\
    \section{求逆元: }\\
    \subsection{式子:}\\
    设数\(x\)的逆元为\(inv[x]\), 数\(p\)为一个质数,则\[inv[x] = (p - p / x) \times inv[p  \%\ x]\]
    \subsection{证明方法:}\\
    我们假设质数\(p\)可以最简的表示为:\(p = a \times x + b\)\\
    显然的,有\(x > b\),因为不然的话可以从\(b\)中再提出一个\(x\)\\
    所以有以下的递推式:\\
    \[p\equiv 0 (mod \ \ p)\]
    \[a \times x + b \equiv 0 (mod \ \  p)\]
    因为我们知道,对于模等式,两边同时加、减、乘一个式子依然成立。\\
    所以两边同时乘以\(inv[x]\cdot inv[b]\)后:\\
    \[a\times inv[b] + inv[x] \equiv 0 (mod \ \ p)\]
    为什么就不用详细解释,将式子展开,因为 \\ \(x \times inv[x] \equiv 0 (mod \ \ p)\) \\
    然后:
    \[inv[x] \equiv -a \times inv[b] (mod \ \ p)\]
    \[p \times inv[b] + inv[x] \equiv p \tiems inv[b] - a\times inv[b] (mod \ \ p)\]
    因为\(p \times inv[b] \% p == 0\)\\
    接下来省略后面的\(mod\)符号\\
    \[inv[x]\equiv p\times inv[b] - a\times inv[b]\]
    \[inv[x]\equiv (p - a) \times inv[b]\]
    此时我们应该把不知道的\(a, b\)去掉才好求的。\\
    再回到原来的式子:\(p=a\times r+b\)\\
    因为\(r > b\),所以由计算机的整数相除法则,可以知道\(p/r = a\),相应的\(p\%r=b\)\\
    所以就可以得到最终的式子啦。\\
    \[inv[x] = (p\ \ -\ \ p\div r) \times inv[p\% r]\]
  \end{itemize}
\end{CJK}
\end{document}
